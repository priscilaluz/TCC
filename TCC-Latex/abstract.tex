The increasing of technologies embedded in mobile devices (e.g., tablets, smartphones) and its use increasingly constant in carrying out various activities of daily life (e.g. paying bills, internet access, access to maps, recording/playback of audio and video) has boosted the growth of applications that use features available on mobile devices (e.g. Bluetooth, Wi-fi, GPS) and the environment in which it is located (e.g. presence and temperature sensors, location, date and time, etc.). The goal is to provide users with services that bring convenience and practicality in their daily activities with minimal explicit interaction. Computational solutions in this area are called context sensitive applications and can add great applicability in users daily life. In this study-based context, perceiving the need to maintain users of public transportation well informed about major transportation services available and, above all, of what is happening on the main roads of a particular city, this paper proposes the creation of a context-aware mobile system that, through collaboration of user, can propagate information related to current traffic conditions and the main services of public transportation available (e.g. location of buses and stops, routes, schedules, route names and etc.), letting users aware in, virtually, real-time about traffic events (e.g. accidents, traffic jams, works and etc.) that are hindering the mobility of routes that possibly they use to carry out their activities, either go to work, school, shopping or simply for recreation. The focus is to improve the information provided to users of public transportation, based on information provided by other individuals.

\begin{keywords}
Ubiquitous Computing, Crowdsourcing, FindBus, context sensitive.
\end{keywords}
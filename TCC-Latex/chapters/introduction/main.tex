\chapter{Introdução}
\label{ch:introduction}

\section{Motivação}
\label{sc:motivacao}

Na atualidade as pessoas estão cercadas por diversas informações e distrações que terminam por promover, uma forma de pensar imediatista e superficial. Por isso, torna-se cada vez mais dificultoso incentivar os alunos ao aprendizado do conteúdo através das aulas tradicionais, nas quais o professor expõe o assunto e os alunos são avaliados por um método clássico. \citep{desafioeducadores}

Com o objetivo de se evitar turmas com alunos dispersos e aulas monótonas, surgiu a gamificação na educação. A aplicação desse modelo visou solucionar o problema sistémico em que o estudante decora a matéria, não aprende o ensino de forma lógica e consequentemente não se interessa pela busca de um conhecimento aprofundado. \citep{gamificacaoaplicada}

É notável que jogos empolgam e motivam as pessoas que ali estão envolvidas, e justamente em função disso os conceitos e dinâmicas da técnica da gamificação foram popularizados e aplicados nos ambientes educacionais. Inclusive, uma das grandes vantagens de se utilizar essa técnica é porque ela possibilita aos estudantes a assimilação do conhecimento através de múltiplas tentativas, competições e interações com jogadores.\citep{gamificacaoaplicada}

Segundo o artigo User-Enjoyable Learning Environment Based on Gamification \citep{ieee2015}, a gamificação é o uso de elementos de jogos em outros contexto com o objetivo de melhorar o envolvimento das pessoas que participam da atividade. Ademais, existem estudos que testificam que o uso desse método está diametralmente ligado ao aumento do rendimento pessoal dos alunos, como também a participação deles em atividades complementares. Após tudo delineado, conclui-se, a gamificação pode ser utilizado em ambientes educacionais para motivar e auxiliar os alunos a aprenderem os conteúdos de forma mais efetiva \citep{extensiblegamification}.

\section{Problema}
\label{sc:problema}

Nos dias atuais, existem diversos sites que apoiam métodos tradicionais de ensino, alguns exemplos são: Moodle, Sakai e Dokeos. No entanto, quando se trata de plataformas que permitam ao professor criar cursos e atividades online de temas referentes à sua matéria de forma intuitiva e dinâmica, mostra-se muito escasso o número de páginas disponíveis na internet. Em vista disso, o docente termina por permanecer limitado à aplicação dos métodos da gamificação de forma presencial na sala de aula. 

Diante de tais fatos e dos notáveis pontos positivos associados ao uso da gamificação, observou-se a relevância de desenvolver uma plataforma que permita ao educador inserir aulas utilizando elementos de jogos, tais como competição, cooperação, ranking, níveis de dificuldade, risco, regras e objetivos bem definidos para despertar o interesse dos estudantes.

\section{Objetivos}
\label{sc:objetivo}

Dessa forma, o trabalho tem por objetivo desenvolver uma plataforma dinâmica e flexível que permita a inserção de cursos sobre assuntos variados, visando despertar o interesse dos educandos por intermédio dos conceitos de gamificação.

Na plataforma educacional, o professor criará uma sequência de aulas organizadas por níveis de conhecimento, sendo que ao aluno somente será permitido passar para um nível mais elevado após atingir determinada quantidade de pontos na atual fase. Além disso, cada etapa será composta de um vídeo, texto e/ou imagem do conteúdo da aula, e ainda incluirá um jogo vinculado a este assunto abordado.

Com o propósito de aumentar a curva de aprendizagem do conteúdo, o aluno que participar dos jogos da plataforma, receberá, ao fim do desafio, um relatório com todas as explicações das perguntas constantes na partida. E, ainda, haverá um ranking mostrando os melhores jogadores, os quais podem receber bônus nos próximos desafios, a exemplo de dicas ou tempo extra.

O site disponibilizará na sua versão beta os jogos supracitados, todavia, caso um usuário deseje acrescentar novas atividades utilizando os conceitos de gamificação, a plataforma dará o suporte necessário. O projeto, nesse sentido, será construído de modo flexível e, inclusive, disponibilizará um tutorial orientado a forma de implementação que o desenvolver deverá adotar à inserção de novos jogos.

Conclusivamente, a plataforma educacional baseada nos métodos de gamificação será dinâmica e permitirá o ensino de qualquer conteúdo em um formato divertido e envolvente. 

\section{Metodologia}
\label{sc:metodologia}

O estudo para o desenvolvimento dessa plataforma realizou-se com fundamento em artigos que abordavam três temas principais: 1) aplicação da gamificação na educação; 2) plataformas educacionais; 3) engenharia de software. Por intermédio desses temas, escolheu-se quais práticas da gamificação, as tecnologias e a arquitetura que foram implementadas para desenvolver essa aplicação.

\section{Estrutura do Trabalho}
\label{sc:estruturaDoTrabalho}

Este trabalho está estruturado da seguinte forma:    

\begin{itemize}
\item Capítulo 2: apresenta o referencial teórico mostrando as vantagens das plataformas de gerenciamento de aprendizagem (LMS). Além disso, será exposta as vantagens do uso da gamificação nessas plataforma. Por fim, neste capítulo também serão mostrados os conceitos de plataformas educacionais.
\item Capítulo 3: aponta trabalhos relacionados, plataformas e aplicações que fazem uso da gamificação e podem ser consideradas uma LMS.
\item Capítulo 4: expõe a proposta desse trabalho ao mostrar toda a arquitetura, as regras de negócio, as especificação de requisitos e os diagramas. Ademais, o capítulo exibir as principais telas e funcionalidade do site.
\item Capítulo 5: neste capítulo será feito a validação da relevância do trabalho através de um questionário respondido por usuários do sistema.
\item Capítulo 6: este é o capítulo conclusivo nele é apresentado o resumo do trabalho e indicado os próximos passos que almejam ser atingidos por intermédio do projeto.

\end{itemize}




A engenharia de software é uma disciplina que integra processos, métodos e ferramentas para o desenvolvimento de programas de computador. Para apoiá-la existem sistemas de gerenciamento de ciclo de vida de aplicativos (ALM), que são ferramentas de software que auxiliam na organização e moderação de um projeto ao longo de seu ciclo de vida. Existe um número infinito de possibilidades para definir o ciclo de vida de um projeto, e as ferramentas necessárias para ajudá-la a conclusão. Consequentemente, uma enorme variedade de sistemas de gestão estão disponíveis no mercado, especificamente concebidos de acordo com um número diversificado de metodologias de gestão. No entanto, a maioria dos sistemas são proprietários, bastante especializados e com pouca ou nenhuma capacidade de personalização, tornando muito difícil encontrar um que se encaixam perfeitamente às necessidades do seu projeto.

Um tipo diferente de desenvolvimento de sistemas de software é a Engenharia de Linha de Produtos de Software - ELPS. LPS é uma metodologia para o desenvolvimento de uma diversidade de produtos de software relacionados e sistemas com uso intensivo de software. Durante o desenvolvimento de uma LPS, uma vasta gama de artefatos devem ser criados e mantidos para preservar a consistência do modelo de família durante o desenvolvimento, o e que é importante para controlar a variabilidade SPL e a rastreabilidade entre os artefatos. No entanto, esta é uma tarefa difícil, devido à heterogeneidade dos artefatos desenvolvidos durante engenharia de linha de produto. Manter a rastreabilidade e artefatos atualizados manualmente é um processo passível de erro, demorado e complexo. Utilizar uma ferramenta para apoiar essas atividades é essencial.

Neste trabalho, propomos o Ambiente Construção Integrado de Linha de Produto de Software ( SPLICE ) . Essa é uma ferramenta online de gerenciamento de ciclo de vida que gerencia, de forma automatizada, as atividades da linha de produtos de software. Esta iniciativa pretende apoiar a maior parte das atividades do processo de LPS, como escopo, requisitos, arquitetura, testes, controle de versão, evolução, gestão e práticas ágeis. 

Nós apresentamos um metamodelo leve, que integra o processo de ciclo de vida das Linhas de Produtos com práticas ágeis, a implementação de uma ferramenta que utiliza o metamodelo proposto , e um estudo de caso que reflete a viabilidade e flexibilidade desta solução especialmente para diferentes cenários e processos.


\begin{keywords}
ferramenta, busca
linha de produtos de software, métodos ágeis, LPS , sistema de gerenciamento de ciclo de vida de aplicativos  , ferramenta, metamodelo

\end{keywords}
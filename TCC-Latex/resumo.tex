O crescente aumento das tecnologias embutidas em dispositivos móveis (e.g. smartphones, tablets) e, seu uso cada vez mais constante na realização de diversas atividades do cotidiano (e.g. pagamento de contas, acesso à internet, acesso a mapas, gravação/reprodução de áudio e vídeo) tem alavancado o crescimento de aplicações que utilizam os recursos disponíveis no dispositivo móvel (e.g. Bluetooth, Wi-fi, GPS) e no ambiente em que esse se encontra (e.g. sensores de presença e temperatura, localização, data e hora, etc.). O objetivo é oferecer aos usuários serviços que tragam comodidade e praticidade em suas atividades diárias com o mínimo de interação explicita. Soluções computacionais neste domínio recebem o nome de aplicações sensíveis ao contexto e podem agregar excelentes aplicabilidades no cotidiano dos usuários. Neste contexto, constatando-se, através de estudos realizados, a necessidade de manter os usuários de transporte públicos bem informados acerca dos principais serviços de transportes disponíveis e, sobretudo, do que está acontecendo nas principais vias de uma determinada cidade,  este trabalho propõe a criação de um sistema mobile sensível ao contexto que, através da colaboração do usuário, possa propagar informações relacionadas a atual situação do trânsito e aos principais serviços do transporte público disponíveis (e.g. localização dos ônibus e pontos, rotas, horários, linhas e etc.), fazendo, desta forma, com que seus usuários tomem conhecimento em, praticamente, tempo real dos eventos de trânsito (e.g. acidentes, engarrafamentos, obras e etc.) que estão dificultando a mobilidade das vias que, possivelmente, eles utilizam para realizar suas atividades, seja ir ao trabalho, estudar, fazer compras ou, simplesmente, para o lazer. O foco é melhorar a informação oferecida aos usuários do transporte público, com base em informações fornecidas por outros indivíduos. 


\begin{keywords}
Computação Ubíqua, Computação Sensível ao Contexto, Inteligência Coletiva, \textit{Crowdsourcing}, Sistemas de Transporte Inteligentes, FindBus. 

\end{keywords}
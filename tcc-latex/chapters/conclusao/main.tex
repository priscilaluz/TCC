\chapter{Conclusão e Trabalhos Futuros}
\label{ch:conclusion}

Este capitulo tem o objetivo de mostrar a conclusão das principais fases deste trabalho, mostrando algumas das dificuldades enfrentadas, bem como as possibilidades de trabalhos futuros que podem estender e enriquecer a proposta apresentada.

\section{Conclusão}
\label{sc:conclusao}

Embasando-se no levantamento bibliográfico realizado no presente trabalho, notou-se a necessidade de prover aos usuários do transporte público informações acerca dos serviços de transporte disponíveis, condições das vias urbanas e, sobretudo, a relação entre estas. Dando-lhes, assim, subsídios suficientes para tomadas de decisões, como por exemplo, qual linha pegar para se chegar a um determinado destino, qual ônibus desta linha utilizar, em que ponto pegar este ônibus e quando chegar neste ponto, bem como a capacidade do autogerenciamento.  

Isto posto, aproveitando-se da grande inclinação da sociedade para o uso de dispositivos móveis e a existências de diversas soluções interessantes utilizando tais dispositivos, desenvolveu-se o FindBus, aplicativo mobile sensível ao contexto de apoio ao transporte público, cujas principais funcionalidades foram mostradas na seção \ref{sc:telas} deste trabalho.
Objetivando analisar se os objetivos que fora especificados neste trabalho, realmente, foram atingidos com o desenvolvimento do aplicativo, realizou-se uma prova de conceito. Através dos resultados obtidos, foi possível concluir que as funcionalidades propostas pelo aplicativo estão de acordo com os objetivos especificados, haja vista que os usuários conseguiram utilizar as principais funcionalidades sem apresentar dificuldades e, sobretudo, avaliaram tais funcionalidades de maneira muito positiva. 

Por meio da realização da prova de conceito foi possível perceber, também, algumas necessidades presentes no aplicativo, como por exemplo, integração com outros meios de transporte, como trem, metrô e Ferry-Boat, muito comum nas grandes cidades. Atualmente, o aplicativo não mostra nenhum tipo de informação a respeito dos serviços de transporte metroviários e aquáticos das cidades em que o mesmo já se encontra em funcionamento, isso para uma grande cidade como pode ser prejudicial, visto que, costumeiramente, a população utiliza tais serviços. 

Ademais, vale ressaltar que algumas dificuldades foram encontradas durante o desenvolvimento da proposta, dentre as quais, conseguir a real localização dos veículos através do aparelho GPS presente neles destaca-se como principal dificuldade, pois essa informação é de posse das prefeituras e de algumas empresas a ela filiada, sendo, na maioria das vezes, um processo muito burocrático consegui-las. Como um dos principais objetivos do aplicativo desenvolvido é, exatamente, fazer com que os usuários consigam saber onde estão os ônibus, bem como visualizar o deslocando destes em, praticamente, tempo real, não conseguir a real localização dos veículos poderia impossibilitar totalmente o desenvolvimento deste trabalho.


\section{Trabalhos Futuros}
\label{sc:trabalhosFuturos}

Conforme fora dito na seção anterior, a proposta desenvolvida não mostra nenhum tipo de informação a respeito dos serviços de transporte metroviários e aquáticos das cidades em funcionamento. Em outras palavras, o aplicativo desenvolvido não mostra a localização, os horários dos serviços de trem, metrô e ferry-boat, bem como não utiliza esses tipos de transporte para recomendar a melhor maneira de um usuário se deslocar de um ponto a outro da cidade. Pensando em uma grande cidade como São Paulo, onde as pessoas costumeiramente utilizam o metrô para se deslocar, isso pode ser visto como um problema. Nesse sentido, sugere-se integrar a proposta desenvolvida aos serviços de transporte metroviários e aquáticos como trabalhos futuros.

Uma das principais funcionalidades do aplicativo desenvolvido é fazer com que as pessoas informem o que está acontecendo no trânsito da cidade para que essa informação seja espalhada rapidamente, de modo a dar a capacidade do autogerenciamento à população. Contudo, isso somente é permitido de dentro do aplicativo, isto é, apenas os usuários da aplicação podem colaborar com a divulgação dessa informação. Atualmente, sabemos que muitas dessas informações são divulgadas nas redes sociais, existindo até perfis de usuários focados em disponibilizar este tipo de informação. 

Associado a isso, sabemos, também, da grande quantidade de pessoas compartilhando informação em redes sociais, como por exemplo, Facebook e Twitter. Isto posto, sugere-se como trabalho futuro integrar a proposta desenvolvida a redes sociais, de modo a coletar informações referentes aos principais acontecimentos atrelados ao trânsito e a mobilidade urbana. Aumentado, assim, a capacidade do aplicativo em manter seus usuários informados dos principais acontecimentos no trânsito e, por conseguinte, recomendar os melhores caminhos para estes chegarem em seus destinos.
